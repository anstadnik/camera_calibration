\chapter{Background}\label{cha:background}

\section{Notation}\label{sec:notation}

\begin{table}[htbp]
	\label{tab:notation}
	\centering
	\begin{tabular}{rl}
		\toprule
		Term                           & Description                                                                                            \\
		\midrule
		\(\mathbf{u} = \begin{pmatrix}
			               u, v, 1
		               \end{pmatrix}^{T}\) & A point in the board coordinate system                                                             \\
		\(\mathbf{x} = \begin{pmatrix}
			               x, y, z, 1
		               \end{pmatrix}^{T}\) & A point in the world coordinate system                                                             \\
		\(R\)                          & A \(3 \times 3\) rotation matrix                                                                       \\
		\(\mathbf{t}\)                 & A \(3 \times 1\) translation vector                                                                    \\
		\(\alpha_x\)                   & Scale factor in the x direction (pixels/mm)                                                            \\
		\(\alpha_y\)                   & Scale factor in the y direction (pixels/mm)                                                            \\
		\(c_x, c_y\)                   & Coordinates of the principal point (image center)                                                      \\
		\(\theta\)                     & Angle between the x and y pixel axes                                                                   \\
		\(f\)                          & Distance from the camera center to the image plane (focal length)                                      \\
		\(f_x, f_y\)                   & Effective focal lengths in the x and y directions                                                      \\
		\(K\)                          & Intrinsic matrix incorporating the scaling, introduced by the focal length                             \\
		\(H\)                          & A \(3 \times 3\) matrix viewing \(z=0\) (see \cref{sub:projection_of_the_points_from_the_scene_plane}) \\
		\(\lambda_n\)                  & Distortion coefficients                                                                                \\
		\bottomrule
	\end{tabular}
	\caption{Notation}
\end{table}

\section{Camera model}\label{sec:camera_model}

In this paper, scene and image points are represented using homogeneous
coordinates. This approach allows to represent many geometric transformations as
linear, which simplifies the mathematical representation of the camera model.

\subsection{Perspective projection}\label{sub:perspective_projection}

The perspective projection is a mapping from a 3D point \(\begin{pmatrix}
	x, y, z
\end{pmatrix}^{T}\)
in the world coordinate
to the 2D coordinate \(\begin{pmatrix}
	u, v
\end{pmatrix}^{T}\)
on the image plane which is distance \(f\) from the center
of projection. It is given by the perspective projection equation:
\[
	\begin{pmatrix}
		u, v
	\end{pmatrix}^{T} = \frac{f}{z} \begin{pmatrix}
		x, y
	\end{pmatrix}^{T}.
\]
This equation can be written using the homogeneous coordinates:
\begin{equation} \label{eq:perspective_projection}
	\alpha \begin{pmatrix}
		u \\ v \\ 1
	\end{pmatrix} = \begin{bmatrix}
		f & 0 & 0 & 0 \\
		0 & f & 0 & 0 \\
		0 & 0 & 1 & 0
	\end{bmatrix} \begin{pmatrix}
		x \\ y \\ z \\ 1
	\end{pmatrix},
\end{equation}
where \(\alpha = \sfrac{1}{z}\) is a scale factor.

\subsection{Scene to camera projection}\label{sub:scene_to_camera_projection}

A 3D scene point \(\begin{pmatrix}
	x, y, z
\end{pmatrix}^{T}\) can be projected onto the image plane as
\(R \begin{pmatrix}
	x, y, z
\end{pmatrix}^{T} + \mathbf{t}\), where \(R\) is a \(3 \times 3\) rotation matrix
and \(\mathbf{t}\) is
a \(3 \times 1\) translation vector. Using the homogeneous coordinates, this
can be written as:
\begin{equation}
	\begin{bmatrix}
		R              & \mathbf{t} \\
		\mathbf{0}^{T} & 1
	\end{bmatrix} \begin{pmatrix}
		x \\ y \\ z \\ 1
	\end{pmatrix}.
\end{equation}

\subsection{Camera to image projection}\label{sub:camera_to_image_projection}

To project a point from the camera coordinate system to the image plane, we
need to apply a homography encoding the camera's intrinsic parameters.
This is a \(3 \times 3\) upper-triangular matrix:
\begin{equation}
	\begin{bmatrix}
		\alpha_x & \alpha_x \cot \theta & c_x \\
		0        & \alpha_y \sin \theta & c_y \\
		0        & 0                    & 1
	\end{bmatrix},
\end{equation}

where:
\begin{itemize}
	\item $\alpha_x$ and $\alpha_y$ represent the scale factor of the camera in
	      terms of \sfrac{pixels}{mm} in the x and y directions respectively.
	\item $c_x$ and $c_y$ are the coordinates of the principal point, which is typically the image center.
	\item $\cot \theta$ and $\sin \theta$ are related to the skew coefficient, which measures the angle between the x and y pixel axes. The variable $\theta$ represents this angle.
	      % \item $f$ is the distance from the camera center to the image plane, which is also known as the focal length.
	      % \item $f_x$ and $f_y$ represent the effective focal lengths in the x and y
	      %       directions in pixel units.
	      % \item $k$ represents the skew.
\end{itemize}
For a typical camera, \(\theta = \sfrac{\pi}{2}\) and \(\alpha_x = \alpha_y\)
\cite{hartleyMultipleViewGeometry2004}.

Conventionally, the intrinsic matrix incorporates the scaling, introduced
by the focal length:
\begin{equation}
	K = \begin{bmatrix}
		\alpha_x & \alpha_x \cot \theta & c_x \\
		0        & \alpha_y             & c_y \\
		0        & 0                    & 1
	\end{bmatrix} \begin{bmatrix}
		f & 0 & 0 \\
		0 & f & 0 \\
		0 & 0 & 1
	\end{bmatrix} = \begin{bmatrix}
		f_x & k   & c_x \\
		0   & f_y & c_y \\
		0   & 0   & 1
	\end{bmatrix}.
\end{equation}

By incorporating the assumptions, mentioned previously into the intrinsic matrix,
we can simplify it to:
\begin{equation}
	K = \begin{bmatrix}
		f & 0 & c_x \\
		0 & f & c_y \\
		0 & 0 & 1
	\end{bmatrix}.
\end{equation}

\subsection{Camera matrix}\label{sub:camera_matrix}

The composition of positioning and orienting the camera, projection, and
imaging transformation can be represented by a $3 \times 4$ camera
matrix \citep{scaramuzzaFlexibleTechniqueAccurate2006}. This matrix can be expressed as:

\begin{equation}
	K \begin{bmatrix}
		I_3 \vert \mathbf{0}
	\end{bmatrix} \begin{bmatrix}
		R              & \mathbf{t} \\
		\mathbf{0}^{T} & 1
	\end{bmatrix} = K \begin{bmatrix}
		R \vert \mathbf{t}
	\end{bmatrix},
\end{equation}

Hence, the transformation of a point in the scene by the camera $\mathrm{P}^{3 \times 4}$ can be formulated as:

\begin{equation}
	\alpha(u, v, 1)^{T} = K \begin{bmatrix}
		R \vert \mathbf{t}
	\end{bmatrix} \begin{pmatrix}
		x, y, z, 1
	\end{pmatrix}^{T},
\end{equation}

with $\alpha$ being $1 / z$.

\subsection{Projection of the points from the scene plane}\label{sub:projection_of_the_points_from_the_scene_plane}

When working with the coplanar scene points, we can simplify the projection
by assuming that the scene plane is located at \(z = 0\). In this case, the
projection of the point becomes:
\begin{equation}
	\alpha \begin{pmatrix}
		u \\ v \\ 1
	\end{pmatrix} = K \begin{bmatrix}
		\mathbf{r_1} & \mathbf{r_2} & \mathbf{r_3} & \mathbf{t}
	\end{bmatrix} \begin{pmatrix}
		x \\ y \\ 0 \\ 1
		% \end{pmatrix} = \begin{bmatrix}
		%   \mathbf{p_1} & \mathbf{p_2} & \mathbf{p_4}
		% \end{bmatrix} \begin{pmatrix}
		%   x \\ y \\ 1
		% \end{pmatrix}.
	\end{pmatrix} = K \underbrace{\begin{bmatrix}
			\mathbf{r_1} & \mathbf{r_2} & \mathbf{t}
		\end{bmatrix}}_{H} \begin{pmatrix}
		x \\ y \\ 1
	\end{pmatrix}.
\end{equation}

\subsection{Distortion}\label{sub:distortion}

The distortion of the image is caused by the lens not being perfectly planar.
Typically, the small distortions caused by lens misalignment are ignored
\cite{hartleyMultipleViewGeometry2004}, allowing us to model the distortion as radially symmetric.
Then, the function that maps a point \(\mathbf{u} = \begin{pmatrix}
	u, v, 1
\end{pmatrix}^{T}\) from a retinal plane to the
ray direction in the camera coordinate system is given by:
\begin{equation}
	g(\mathbf{u}) = \begin{pmatrix}
		u, v, \psi(r(\mathbf{u}))
	\end{pmatrix}^{T},
\end{equation}
where \(r(\mathbf{u}) = \sqrt{u^2 + v^2}\) is the radial distance from the
principal point.

\subsubsection{Back-projection using the Division Model}\label{subsub:back_projection_using_the_division_model}

The division model has a good ability to model the distortion of the wide-angle
lenses \citep{fitzgibbonSimultaneousLinearEstimation2001}, and is wildly used \citep{prittsMinimalSolversRectifying2021,
	scaramuzzaFlexibleTechniqueAccurate2006}. The model is defined as:
\begin{equation}
	\psi(r) = 1 + \sum_{n = 1}^{N} \lambda_n r^{2n},
\end{equation}
where \(\lambda_n\) are the distortion coefficients.

The function \(\psi(r)\) is not invertible in general.
Let \(\mathbf{\widehat{x}} = \begin{pmatrix}
	x, y, z
\end{pmatrix}^{T} = \alpha g(\mathbf{u})\) be a ray in the camera coordinate system.

Then,
\begin{align}
	\frac{\mathbf{x}}{z} & =
	\begin{pmatrix}
		\frac{x}{z}, \frac{y}{z}, 1
	\end{pmatrix}^{T}                               \\                                  & =
	\begin{pmatrix}
		\frac{\alpha u}{\alpha \psi(r(\mathbf{u}))},
		\frac{\alpha v}{\alpha \psi(r(\mathbf{u}))},
		1
	\end{pmatrix}^{T}  \\
	                     & = \label{eq:division_derivation1}
	\begin{pmatrix}
		\frac{u}{\psi(r(\mathbf{u}))},
		\frac{v}{\psi(r(\mathbf{u}))},
		1
	\end{pmatrix}^{T}.
\end{align}

From \ref{eq:division_derivation1} we see that
\begin{equation} \label{eq:division_derivation2}
	\begin{cases}
		\frac{x}{z} = \frac{u}{\psi(r(\mathbf{u}))} \\
		\frac{y}{z} = \frac{v}{\psi(r(\mathbf{u}))}
	\end{cases} \implies \\
	\begin{cases}
		u = \frac{x \psi(r(\mathbf{u}))}{z} \\
		v = \frac{y \psi(r(\mathbf{u}))}{z}
	\end{cases}.
\end{equation}

Now, let \(\widehat{r}\) be a root of
\(r(\mathbf{u})
= \sqrt{
	\frac{x * \psi \left( r\right)}{z}^{2} +
	\frac{y * \psi \left( r\right)}{z}^{2}
} = r\).

Then, \(\mathbf{u} = f(\mathbf{x}) =
\frac{\widehat{r}}{r(\mathbf{x})}\mathbf{x}\),
where \(f(\cdot)\) is the inverse of \(g(\cdot)\).

\subsection{Complete projection and backprojection}\label{sub:complete_projection_and_backprojection}

Now, the complete projection and back-projection can be formulated as follows:
\begin{align}
	\alpha \mathbf{u} & = K f(H\mathbf{x})
	\tag{Projection} \label{eq:projection}                     \\
	\alpha \mathbf{x} & = H^{-1} g(K^{-1}\mathbf{u}) \tag{Back
		projection} \label{eq:back_projection}.
\end{align}

