\chapter{Conclusions}\label{cha:conclusions}

In this thesis, we have presented a new approach for feature detection
improvement. We used the feature detector, proposed by
\cite{geigerAutomaticCameraRange2012}, to detect the initial features. Then, we
found the camera calibration for the camera model, proposed by
\cite{scaramuzzaToolboxEasilyCalibrating2006}, and then improved it by
minimizing the reprojection error between the back-projection of the detected
features and the board. Lastly, we used the improved camera calibration to
project the imputed and extended board back into the image and used a binary
classification to find the previously undetected features. We tested the method
on a real-world dataset and showed that it can improve the feature detection.

\section{Future work}\label{sec:future_work}

The method, presented in this thesis can be improved in several ways. First, the
feature detection algorithm can be tweaked to use alternating step sizes while
constructing the board. It would allow to also match other types of the boars,
such as ArUco or AprilTag. Second, the method can be extended to multiple boards
on the same image, and multiple images. It would improve the robustness of the
camera calibration, and reduce the number of detected points on the wrong board
when there is overlap. Lastly, additional improvements can be made to the corner
classification algorithm, specifically to the response function. One thing that
could work is resizing the image, to find the corners of different sizes.

